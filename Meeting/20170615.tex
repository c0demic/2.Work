% Created 2017-06-18 Sun 23:40
\documentclass[11pt]{article}
\usepackage[utf8]{inputenc}
\usepackage[T1]{fontenc}
\usepackage{fixltx2e}
\usepackage{graphicx}
\usepackage{longtable}
\usepackage{float}
\usepackage{wrapfig}
\usepackage{rotating}
\usepackage[normalem]{ulem}
\usepackage{amsmath}
\usepackage{textcomp}
\usepackage{marvosym}
\usepackage{wasysym}
\usepackage{amssymb}
\usepackage{hyperref}
\tolerance=1000
\author{Lixin Yu}
\date{\today}
\title{Minutes of China-Chile Database workshop}
\hypersetup{
  pdfkeywords={},
  pdfsubject={},
  pdfcreator={Emacs 25.1.2 (Org mode 8.2.10)}}
\begin{document}

\maketitle
\tableofcontents

\newpage

Attendees: Mauricio Solar, Mauricio Araya, Humberto Farias, Calos, Matias, Matine
Attendees: Zhong Wang, Wei Wang, Nanyao Lv, Jinhua He, Lei Zhu, Chunhui Shi, Lixin Yu


\section{Contents}
\label{sec-1}
\subsection{Data Access to Alma Data}
\label{sec-1-1}
\begin{itemize}
\item part of open alma data(30TB)(bring by disks copy)
\begin{itemize}
\item all the open fits files are available
\item lack ASDM(cycle 0,1,2 are available, cycle 3 are public but not sure if all cycle 3 are obtained)
\end{itemize}
\item data process
\begin{itemize}
\item access through vpn(jupyter)
\item jupyter hub through docker
\item lack of astromer testing jupyter
\end{itemize}
\item direct hpc computing
\begin{itemize}
\item simulation
\item need to program with C
\item software are provided by Huawei and might be changed further
\end{itemize}
\item Jovial
\begin{itemize}
\item no public IP, through vpn, need account
\item HPC is not supported through GUI
\item quota can be imposed through BCM
\end{itemize}
\item IP
\begin{itemize}
\item for service not public needed
\end{itemize}
\item VM
\begin{itemize}
\item 2 machine for vm, might need to move one from computing node to vm
\begin{itemize}
\item now everything is okay
\end{itemize}
\end{itemize}
\item Apache Spark
\begin{itemize}
\item the part we are going to join
\item june 29 10:00 spanish talk about archetecture
\end{itemize}
\item access
\begin{itemize}
\item public: VO
\item private: HPC, Jovial
\item management software: ibmc, neteco, smtp, lustre, ovirt, smtp, LDAP,
\end{itemize}
\item network
\begin{itemize}
\item calan: no reuna, problem in ISP(not solved yet)
\begin{itemize}
\item we are informed that calan would have an update for internet but we don't
\item calan have a fiber but the property is catolica(but we can pay some money and make a big improvement to the internet bandwidth and share the bandwidth)
\end{itemize}
\end{itemize}
\item Data
\begin{itemize}
\item 4.5TB data available
\item ADQL similar to SQL
\item first VO to deal with big data like ALMA
\item only qa data(fits) and adsm file,
\item metadata is about 1\% of the total meta data extract asdm
\end{itemize}
\end{itemize}

\subsection{Service:}
\label{sec-1-2}
\begin{itemize}
\item character: much faster for south american astronomy community
\item provide btter stamp image
\begin{itemize}
\item 3D
\item better than compressed icon image?
\end{itemize}
\end{itemize}

\subsection{CASSACA and ChiVO:}
\label{sec-1-3}
\begin{itemize}
\item CASSACA is more customized, ChiVO is like that
\item ChiVO can provide the alma data, CASSACA is not authorized, CASSACA need to ask permissions
\item ? Can Cassaca host other data that ChiVO host(gemini or other things)
\item official name for chile-china astronomy data center and logo
\item 
\end{itemize}

\subsection{Security and network Issue}
\label{sec-1-4}
\begin{itemize}
\item no https only http
\item user and password
\item user restriction
\begin{itemize}
\item download bandwidth(put restriction on downloading site)
\end{itemize}
\item lustre
\begin{itemize}
\item we need to update lustre, if we update the lustre, the website would be down for 2 weeks
\item now we can not update lustre
\item we need help from huawei in china(??)
\begin{itemize}
\item can't get connection with Huawei people in China
\item no more info from intel after they ask the log
\end{itemize}
\end{itemize}
\item security
\begin{itemize}
\item virus?
\item hack?
\begin{itemize}
\item port attack
\item from china
\end{itemize}
\item carlos is making the webpage more secure
\item no way to recover from attack in short time
\item add ssl is quick and should be done after the development
\end{itemize}
\item casa install
\begin{itemize}
\item installed by each person
\item no need to run casa on cluster if you don't run MPI, only the memory is large but if there are more people the memory is also low
\item GUI can not work if network speed is slow
\item 
\end{itemize}
\item archive update
\begin{itemize}
\item day by day: program will do it as frequently as  you want
\item fits and other: not sure how frequently will be updated
\end{itemize}
\item jovial (service)
\begin{itemize}
\item not a library, not a GUI(aladin or topcat\ldots{})
\item jupyter extention
\item jupyter hub
\item you can repeat the procedure with the paper
\begin{itemize}
\item starlink
\item astropy
\item casa
\end{itemize}
\item library
\begin{itemize}
\item acalib
\item \ldots{}
\end{itemize}
\end{itemize}

\item hardware
\begin{itemize}
\item move one computing node to vm
\end{itemize}
\end{itemize}
\subsection{CASSACA VO}
\label{sec-1-5}
\begin{itemize}
\item stamp image
\begin{itemize}
\item uv fits file
\end{itemize}
\item service
\begin{itemize}
\item normal service
\begin{itemize}
\item help desk or faq
\end{itemize}
\item expert service
\begin{itemize}
\item 3d/ reproduce data
\end{itemize}
\end{itemize}
\item log information
\end{itemize}
% Emacs 25.1.2 (Org mode 8.2.10)
\end{document}